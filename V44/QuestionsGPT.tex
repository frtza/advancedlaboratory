\documentclass[a4paper,12pt]{article}

% Packages
\usepackage[utf8]{inputenc} % For encoding
\usepackage{amsmath} % For mathematical symbols and equations
\usepackage{amssymb} % For additional mathematical symbols
\usepackage{graphicx} % For including graphics
\usepackage{geometry} % For page layout
\geometry{margin=1in} % Setting margins
\usepackage{hyperref} % For hyperlinks

\title{Basics of X-ray Reflectometry}
\author{}
\date{}

\begin{document}

\maketitle

\section*{Experimental Basics of X-ray Reflectometry}

\subsection*{1. How is X-ray Radiation Produced in an X-ray Tube?}
X-ray radiation is generated when high-energy electrons, accelerated by a voltage difference, strike a metal target (anode) inside an X-ray tube. This process involves:
\begin{itemize}
    \item \textbf{Bremsstrahlung Radiation}: The deceleration of electrons due to the Coulomb field of the atomic nuclei in the target produces continuous X-ray radiation.
    \item \textbf{Characteristic X-rays}: Electrons may displace inner-shell electrons in the target atoms, and when these vacancies are filled by higher-energy electrons, X-rays with specific energies characteristic of the target material are emitted.
\end{itemize}
The intensity and wavelength of the X-rays depend on the accelerating voltage and the material of the anode, with common choices including copper or molybdenum.

\subsection*{2. What is a Göbel Mirror and How Does it Work?}
A Göbel mirror is a device that uses a multilayer coating with alternating layers of high and low electron densities to focus or collimate X-ray beams. Through total external reflection, it directs the X-ray beam along a precise path, enhancing beam intensity and focus, which improves signal quality in reflectometry experiments.

\subsection*{3. How is the Wave Vector Transfer Defined in an X-ray Reflectometry Experiment?}
The wave vector transfer \( q \) (or momentum transfer) quantifies the change in the X-ray wave vector during reflection or scattering and is defined as:
\[
q = k_f - k_i
\]
where \( k_i \) and \( k_f \) are the incident and reflected wave vectors, respectively. In reflectometry, this is often simplified in terms of the angle of incidence \( \theta \):
\[
q_z = \frac{4\pi}{\lambda} \sin \theta
\]
where \( q_z \) is the perpendicular component of \( q \), and \( \lambda \) is the wavelength of the X-rays. The wave vector transfer provides information about the spatial structure of the layer, with larger values of \( q \) probing finer structural details.

\subsection*{4. What are Fresnel’s Formulae and What Do They Describe?}
Fresnel’s formulae describe the reflection and transmission of electromagnetic waves at an interface between two media with different refractive indices. They determine the reflectivity \( R \) as a function of the angle of incidence and the refractive indices of the media.

For X-rays, the refractive index \( n \) in materials is slightly less than 1, which leads to total external reflection at small angles. Fresnel’s equations can thus be adapted to describe the reflection of X-rays off surfaces at near-grazing incidences, essential in X-ray reflectivity studies.

\subsection*{5. What are Kiessig Oscillations and How Do They Occur?}
Kiessig oscillations are interference fringes observed in the reflectivity curve of a thin film when X-rays are reflected. These oscillations result from the constructive and destructive interference between X-rays reflected from the top and bottom interfaces of the film.

The periodicity of the oscillations in the reflectivity curve is related to the film's thickness \( d \) and is given approximately by:
\[
\Delta q_z = \frac{2\pi}{d}
\]
where \( \Delta q_z \) is the spacing between oscillations in reciprocal space. Measuring these oscillations allows for the determination of layer thickness.

\subsection*{6. How Does the Parratt Algorithm Work (Qualitatively)?}
The Parratt algorithm is a recursive method used to calculate the reflectivity of X-rays at multilayer interfaces. It involves:
\begin{itemize}
    \item Dividing the material into discrete layers, each with its own refractive index.
    \item Calculating the reflection and transmission at each interface, considering phase changes and interference effects.
    \item Combining the contributions of each layer iteratively to obtain the total reflectivity.
\end{itemize}
The algorithm is particularly useful for multi-layer structures where simple analytical solutions are impractical.

\subsection*{7. How Must the Parratt Algorithm Be Modified to Apply to Rough Surfaces?}
For rough surfaces, the Parratt algorithm needs to account for the decreased coherence of reflected waves due to surface irregularities. This is often done by:
\begin{itemize}
    \item Incorporating a \textbf{Debye-Waller factor} or \textbf{roughness factor} that modifies the reflectivity at each interface.
    \item Modeling the roughness as a Gaussian distribution, which reduces the reflectivity based on the standard deviation of the surface height \( \sigma \), typically by a term \( e^{-(q_z \sigma)^2} \) that dampens the reflectivity at higher angles (larger \( q \) values).
\end{itemize}
This adjustment allows the Parratt algorithm to provide more accurate predictions of reflectivity for surfaces with significant roughness.

\end{document}
