\section{Diskussion}

Zusammenfassend lässt sich das Analogieexperiment als erfolgreich bewerten. Die dazu vorgenommene
Justierung der Bauteile läuft ohne große Schwierigkeiten ab, auch die Feineinstellung und Funktionsweise
der Apparatur scheint keine besonderen Ansprüche an Genauigkeit zu haben. Dagegen ist zu erwarten, dass
eine Implementierung tatsächlicher Quantenkryptographie mit einzelnen photonischen Zuständen deutlich
sensibler gegenüber solchen Störeinflüssen wäre.

Bei der Durchführung der Messung selbst tritt ebenfalls nichts Unerwartetes auf. Nach sorgfältigem
Testen werden alle Signale den Erwartungen entsprechend übertragen, sodass Nachrichten problemlos
versendet werden können. Ein Punkt, der leicht übersehen werden könnte, ist die korrekte Ausrichtung
der Halbwellenplatten, deren Skalen immer in Richtung des jeweiligen Lasers oder Detektors zeigen
sollten. Ansonsten können gekippte Bits auftreten, obwohl die Basen übereinstimmen.

Anhand der Analyse der Wahrscheinlichkeitsverteilungen wird auch für den Abhörtest klar, dass mit
hoher Sicherheit ein Lauscher zwischen den kommunizierenden Parteien verbaut ist. Die dazu aufgenommene
Fehlerrate liegt im Bereich der Erwartungswerte und ist dadurch wahrscheinlich nicht auf inkorrekte
Anwendung der Apparatur zurückzuführen.
