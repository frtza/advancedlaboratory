\section{Diskussion}

Der folgende Abschnitt fasst die ermittelten Ergebnisse zusammen und bewertet diese. Allgemein lässt sich feststellen, dass diese Größen
im Rahmen der Messunsicherheit plausible Werte aufweisen.

\subsection*{Reichweite}

Die für $E_\alpha = \qty{5.486}{\mega\electronvolt}$ und Normaldruck numerisch bestimmte Reichweite in Luft
\begin{equation*}
	R_\alpha = \qty{6.9}{\centi\meter}
\end{equation*}
liegt wie erwartet zwischen den Werten $\qty{0.5}{\centi\meter}$ und $\qty{9.5}{\centi\meter}$ für $\qty{1}{\mega\electronvolt}$ und
$\qty{10}{\mega\electronvolt}$ Energien aus \cite{Kolanoski_2007}. Da für die numerische Rechnung allerdings die klassische Energieverlustformel
\eqref{eqn:Bethe} verwendet wird, ist zu erwarten, dass der tatsächliche Wert $R_\alpha$ etwas geringer als das hier angegebene Ergebnis ausfällt.

Indem Luft als ideales Gas angenommen wird, folgt aus $pV = nRT$ der Zusammenhang $p \propto V^{-1} \propto \rho \propto N$
und damit $R\alpha \propto p^{-1}$ als Abhängigkeit zur Abschätzung der Relevanz der Abschwächung bei veränderlichem Druck.

\subsection*{Foliendicke}

Mit einem realen Wert von $\qty{2}{\micro\meter}$ liegt die ermittelte Foliendicke
\begin{equation*}
	d = \qty{5.1+-0.4}{\micro\meter}
\end{equation*}
zwar in der richtigen Größenordnung, entspricht allerdings trotzdem einer $\qty{155+-20}{\percent}$ Abweichung. Dazu kommen einige mögliche
Ursachen in Frage. Zunächst wird der linear verlaufende Abschnitt der Messung nach Augenmaß abgeschätzt. Die dabei ausgeschlossenen Werte
flachen wahrscheinlich durch Rauschen ab, könnten aber auch die untere Asymptote der kumulierten Verteilungsfunktion angeben. Den abflachenden
Verlauf gegen sehr geringen Druck fällt ebenfalls aus, sodass die Annahme von $U_\alpha$ als halbes Maximum fragwürdig ist. Zuletzt sollte hier
noch angemerkt werden, dass bei der Messung die verstärkten Spannungspulse am Detektor teilweise starke Schwankungen aufzeigen und die maximale
Amplitude daher nicht gut abgelesen werden konnte. Letzterer Punkt sollte jedoch vor allem zu einer größeren Unsicherheit und nicht zu festen
Abweichungen der Messergebnisse führen.

\subsection*{Streuwinkelabhänigkeit}

Beim Überprüfen der $C \propto \sin^{-4}(\theta / 2)$ Proportionalität zeigt der Fit an das Plateau die beste Abschätzung der Winkelverschiebung.
Für größere Winkel verläuft die angepasste Funktion jedoch steiler als die Messung. Insgesamt lässt sich die gesuchte Abhängigkeit durch die
gewählte Parametrisierung also nicht zufriedenstellend an alle Datenpunkte nähern. Dies ist ein durchaus erwartetes Ergebnis, da der
Teilchenstrahl nicht wie in der Herleitung der Streuformel angenommen eine Punktquelle beschreibt, sondern eine signifikante Breite hat. Um
eine korrekte Beschreibung zu erhalten, müsste also für jede Winkeleinstellung über ein gewisses Winkelintervall integriert werden, um die
Alphastrahlung zu modellieren, die den Detektor erreicht. Mögliche Lösungsansätzte sind eine engere Blende oder ein längerer Detektorarm, was
aber wiederum die Anzahl der aufgezeichneten Impulse drastisch reduzieren würde und so die Statistik verschlechtert.
