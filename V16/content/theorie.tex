\section[Theorie]{Theorie \textnormal{\cite{rutherford}}}

Der Rutherford-Streuungsversuch untersucht die Wechselwirkung von Alphateilchen mit Materie. Um diese Wechselwirkungen zu verstehen, ist es wichtig, zunächst zu erklären, wie Alphateilchen erzeugt werden können. Alphateilchen sind positiv geladene Heliumkerne, die bei der radioaktiven $\alpha$-Emission entstehen. Dieser Prozess tritt auf, wenn instabile schwere Atomkerne, wie Uran ($^{238}\mathrm{U}$) oder Americium ($^{241}\mathrm{Am}$), zerfallen. Bei diesem Zerfall wird ein Alphateilchen emittiert, das aus zwei Protonen und zwei Neutronen besteht. Die Energie der Alphateilchen hängt vom verwendeten Isotop ab, in der Regel beträgt sie mehrere MeV.

Nachdem die Alphateilchen durch einen radioaktiven Zerfall erzeugt wurden, können sie in einem Experiment auf ein Targetmaterial, wie beispielsweise eine dünne Goldfolie, gerichtet werden. Beim Durchgang durch die Materie kann es zu zwei wesentlichen Wechselwirkungen kommen: der Wechselwirkung mit den Elektronen der Atomhülle und der Wechselwirkung mit den Atomkernen.

\subsection*{Wechselwirkung mit den Hüllenelektronen}
Die erste Wechselwirkung erfolgt zwischen den Alphateilchen und den Elektronen in der Atomhülle. Diese Wechselwirkung führt zu einem Energieverlust der Alphateilchen durch Anregung oder Ionisation der Elektronen. Allerdings wird aufgrund der weitaus größeren Masse der Alphateilchen im Vergleich zu den Elektronen keine signifikante Ablenkung der Alphateilchen durch die Elektronen erwartet. Der Energieverlust, den die Alphateilchen während ihres Durchgangs durch die Materie erfahren, wird durch die Bethe-Bloch-Gleichung beschrieben. Diese lautet:

\[
-\frac{\mathrm{d}E}{\mathrm{d}x} = \frac{4 \pi e^4 z^2 N Z}{m_0 v^2 (4 \pi \epsilon_0)^2} \ln \left(\frac{2 m_0 v^2}{I}\right),
\]

wobei in dieser Gleichung $N$ die Atomdichte des Materials beschreibt, also die Anzahl der Atome pro Volumeneinheit. Die Größe $m_0$ ist die Ruhemasse des Elektrons, und $v$ stellt die Geschwindigkeit der Alphateilchen dar. Der Parameter $Z$ bezeichnet die Kernladungszahl des Targetmaterials, und $I$ ist die mittlere Ionisationsenergie des Materials, die die Energie angibt, die erforderlich ist, um ein Elektron aus seiner Bindung im Atom zu lösen. Schließlich beschreibt $e$ die Elementarladung eines Protons, und $\epsilon_0$ ist die elektrische Feldkonstante.

Diese Gleichung beschreibt den kontinuierlichen Energieverlust der Alphateilchen, der durch die Wechselwirkung mit den Elektronen der Atome verursacht wird. Der Verlust hängt dabei von der Geschwindigkeit der Alphateilchen ab, wobei langsame Teilchen einen höheren Energieverlust pro Längeneinheit aufweisen.

\subsection*{Wechselwirkung mit den Atomkernen}
Die zweite Wechselwirkung tritt auf, wenn die Alphateilchen mit den positiv geladenen Atomkernen des Targetmaterials kollidieren. Diese Wechselwirkung ist eine Coulomb-Streuung, die aufgrund der elektrostatischen Anziehung zwischen der positiven Ladung der Alphateilchen und der positiven Ladung der Atomkerne entsteht. Der Streuwinkel $\Theta$ der Alphateilchen kann dabei mit der Rutherford-Streuformel berechnet werden:

\[
\frac{\mathrm{d}\sigma}{\mathrm{d}\Omega} = \frac{1}{(4 \pi \epsilon_0)^2} \left( \frac{z Z e^2}{2 E_\alpha} \right)^2 \frac{1}{\sin^4\left(\frac{\Theta}{2}\right)}\; .
\]

In dieser Gleichung bezeichnet $z$ die Ladungszahl des Alphateilchens (die immer 2 ist, da es aus zwei Protonen besteht), während $Z$ die Kernladungszahl des Targetmaterials ist. Die Größe $e$ stellt wieder die Elementarladung dar, und $E_\alpha$ ist die kinetische Energie der Alphateilchen. Der Streuwinkel $\Theta$ gibt die Richtungsänderung der Alphateilchen nach der Kollision mit dem Atomkern an. 

Der differentielle Wirkungsquerschnitt $\frac{\mathrm{d}\sigma}{\mathrm{d}\Omega}$ beschreibt die Intensitätsverteilung der gestreuten Alphateilchen im Raumwinkel $\mathrm{d}\Omega$, was bedeutet, dass er angibt, wie die Alphateilchen in verschiedene Richtungen gestreut werden. Die Rutherford-Streuformel zeigt, dass die Wahrscheinlichkeit für eine Streuung bei kleinen Winkeln am größten ist und mit zunehmendem Streuwinkel stark abnimmt.

\subsection*{Vernachlässigung der Mehrfachstreuung}
In diesem Experiment wird in der Regel nach der ersten Bornschen Näherung gearbeitet, was bedeutet, dass Mehrfachstreuungen der Alphateilchen vernachlässigt werden. Dies ist eine gültige Annahme, wenn die Targetfolie dünn genug ist und die Alphateilchen nur einmal gestreut werden. Daher können die Alphateilchen als unabhängig voneinander betrachtet werden, und ihre Streuung kann mit der oben genannten Rutherford-Streuformel genau beschrieben werden.

Zusammenfassend beschreibt der Rutherford-Streuversuch die Wechselwirkung von Alphateilchen mit der Materie, wobei insbesondere die Streuung durch Coulomb-Wechselwirkung mit den Atomkernen und die kontinuierliche Energieabgabe durch Wechselwirkungen mit den Elektronen berücksichtigt werden.
