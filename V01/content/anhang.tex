\section*{Anhang}

\addcontentsline{toc}{section}{Anhang}

\centering

\null\vfill
\begin{table}[H]
	\centering
	\caption{Messwerte zu Verzögerungszeit und Zählrate. Aus poissonverteilter Zählung wird für
			 den Fehler die Quadratwurzel der Gesamtzahl gebildet und dann durch das Messintervall
			 $\qty{10}{\second}$ dividiert.}
	    \begin{tabular}{S[table-format=2.0] S[table-format=3.0]}
        \toprule
        {$t \mathbin{/} \unit{\nano\second}$} & {$N$} \\
        \midrule
        -30 &   0 \\
        -29 &   0 \\
        -28 &   0 \\
        -27 &   0 \\
        -26 &   0 \\
        -25 &   0 \\
        -24 &   0 \\
        -23 &   0 \\
        -22 &   0 \\
        -21 &   0 \\
        -20 &   0 \\
        -19 &   0 \\
        -18 &   0 \\
        -17 &   0 \\
        -16 &   0 \\
        -15 &   0 \\
        -14 &   0 \\
        -13 &   0 \\
        -12 &   0 \\
        -11 &   1 \\
        -10 &   2 \\
         -9 &   5 \\
         -8 &   8 \\
         -7 &   8 \\
         -6 &   9 \\
         -5 &  10 \\
         -4 &  13 \\
         -3 &  14 \\
         -2 &  16 \\
         -1 &  20 \\
          0 &  20 \\
          1 &  18 \\
          2 &  22 \\
          3 &  18 \\
          4 &  20 \\
          5 &  20 \\
          6 &  16 \\
          7 &  19 \\
          8 &  14 \\
          9 &  15 \\
         10 &  11 \\
         11 &  10 \\
         12 &  10 \\
         13 &   6 \\
         14 &   5 \\
         15 &   4 \\
         16 &   1 \\
         17 &   1 \\
         18 &   0 \\
         19 &   0 \\
         20 &   1 \\
         21 &   0 \\
         22 &   0 \\
         23 &   0 \\
         24 &   0 \\
         25 &   0 \\
         26 &   0 \\
         27 &   0 \\
         28 &   0 \\
         29 &   0 \\
         30 &   0 \\
        \bottomrule
    \end{tabular}

	\label{tab:delay}
\end{table}
\vfill\null

\newpage

\null\vfill
\begin{table}[H]
	\centering
	\caption{Messwerte zu Kanalindex, Zeitintervall und Zählrate. Einzelne Einträge entsprechen jeweils einer
			 Laufzeit von $\qty{10}{\second}$ am Doppelpulsgenerator.}
	    \begin{tabular}{S[table-format=3.0] S[table-format=1.1]}
        \toprule
        {$K$} & {$t \mathbin{/} \unit{\micro\second}$} \\
        \midrule
          4 & 0.5 \\
         16 & 1.0 \\
         38 & 2.0 \\
         61 & 3.0 \\
         84 & 4.0 \\
        107 & 5.0 \\
        129 & 6.0 \\
        152 & 7.0 \\
        175 & 8.0 \\
        197 & 9.0 \\
        218 & 9.9 \\
        \bottomrule
    \end{tabular}

	\label{tab:calibration}
\end{table}
\vfill\null

\newpage

\null\vfill
\begin{table}[H]
	\centering
	\caption{Zählraten sortiert nach unten und rechts mit aufsteigender Kanalnummer. In der Messzeit
			 $\protect\qty{154196}{\second}$ wurden $\protect\num{2945138}$ Startpulse und
			 $\protect\input{build/nstopp.tex}$ Stopppulse registriert. Die nachfolgenden Werte beziehen
			 sich auf dieselbe Messzeit.}
	\noindent\makebox[\linewidth]{
	    \sisetup{table-format=2.0}
    \begin{tabular}{S S S S S S S S S S S S S S S S}
        \toprule
        \multicolumn{16}{c}{$N$} \\
        \midrule
         0 & 51 & 44 & 35 & 21 & 13 & 14 &  9 & 10 &  5 &  5 &  4 &  3 &  0 &  0 &  0 \\
         0 & 51 & 44 & 33 & 23 & 20 & 13 & 11 &  8 &  7 &  5 &  5 &  3 &  0 &  0 &  0 \\
         0 & 53 & 40 & 19 & 30 & 20 & 12 &  5 &  6 &  5 &  6 &  2 &  2 &  0 &  0 &  0 \\
        17 & 50 & 37 & 34 & 21 & 13 & 17 &  7 &  5 &  1 &  3 &  6 &  1 &  0 &  0 &  0 \\
        67 & 54 & 35 & 33 & 18 & 12 & 12 &  7 &  5 &  3 &  3 &  2 &  3 &  0 &  0 &  0 \\
        62 & 50 & 41 & 34 & 17 & 15 & 10 &  6 &  6 &  8 &  5 &  4 &  3 &  0 &  0 &  0 \\
        50 & 57 & 31 & 21 & 24 & 17 & 11 & 11 &  4 &  5 &  3 &  6 &  0 &  0 &  0 &  0 \\
        51 & 40 & 33 & 22 & 22 & 10 & 11 & 10 & 11 &  5 &  1 &  2 &  3 &  0 &  0 &  0 \\
        71 & 43 & 37 & 26 & 19 & 19 &  9 &  9 &  4 &  4 &  3 &  2 &  4 &  0 &  0 &  0 \\
        56 & 40 & 45 & 24 & 21 & 10 &  9 &  7 &  3 &  4 &  6 &  1 &  2 &  0 &  0 &  0 \\
        60 & 40 & 38 & 31 & 17 & 14 & 12 &  6 &  8 &  3 &  3 &  5 &  0 &  0 &  0 &  0 \\
        62 & 49 & 41 & 32 & 17 & 10 & 13 &  8 &  3 &  8 &  2 &  3 &  1 &  0 &  0 &  0 \\
        73 & 42 & 34 & 26 & 16 & 11 &  8 &  7 &  4 &  8 &  4 &  0 &  1 &  0 &  0 &  0 \\
        71 & 58 & 31 & 25 & 23 & 12 & 10 &  8 &  7 &  5 &  3 &  4 &  3 &  0 &  0 &  0 \\
        71 & 40 & 25 & 28 & 17 & 24 & 10 &  2 &  7 &  2 &  3 &  2 &  3 &  0 &  0 &  0 \\
        73 & 45 & 37 & 24 & 13 &  7 & 13 &  8 &  3 &  4 &  3 &  1 &  2 &  0 &  0 &  0 \\
        45 & 56 & 39 & 21 & 21 &  8 & 13 & 13 &  3 &  4 &  3 &  2 &  3 &  0 &  0 &  0 \\
        65 & 41 & 32 & 26 & 16 & 16 & 14 &  3 &  3 &  1 &  1 &  2 &  3 &  0 &  0 &  0 \\
        77 & 43 & 29 & 17 & 19 & 13 & 13 &  5 &  7 &  3 &  4 &  4 &  1 &  0 &  0 &  0 \\
        75 & 54 & 29 & 24 & 20 & 11 & 13 & 10 &  6 &  0 &  4 &  4 &  3 &  0 &  0 &  0 \\
        98 & 48 & 31 & 21 & 14 & 21 &  7 &  7 &  7 &  4 &  3 &  3 &  5 &  0 &  0 &  0 \\
        95 & 37 & 35 & 23 & 23 & 13 &  6 &  8 &  5 &  5 &  3 &  4 &  1 &  0 &  0 &  0 \\
        71 & 38 & 32 & 24 & 14 &  8 &  6 &  5 &  6 &  1 &  1 &  2 &  1 &  0 &  0 &  0 \\
        61 & 46 & 31 & 28 & 16 & 10 &  9 &  4 &  4 &  4 &  3 &  2 &  2 &  0 &  0 &  0 \\
        54 & 37 & 31 & 24 & 15 & 13 & 11 &  9 &  8 &  6 &  3 &  2 &  3 &  0 &  0 &  0 \\
        80 & 46 & 31 & 15 & 11 &  8 &  6 &  4 &  6 &  6 &  5 &  3 &  1 &  0 &  0 &  0 \\
        56 & 44 & 28 & 20 & 21 & 17 & 10 & 11 &  4 &  5 &  1 &  0 &  2 &  0 &  0 &  0 \\
        53 & 30 & 36 & 14 & 18 & 13 & 11 & 10 &  6 &  4 &  1 &  3 &  7 &  0 &  0 &  0 \\
        48 & 51 & 29 & 22 & 14 & 10 &  7 &  3 &  5 &  4 &  5 &  4 &  3 &  0 &  0 &  0 \\
        62 & 41 & 25 & 18 & 14 &  9 & 11 &  6 & 10 &  8 &  4 &  4 &  0 &  0 &  0 &  0 \\
        48 & 44 & 27 & 21 & 15 & 12 & 10 & 10 &  4 &  4 &  2 &  5 &  2 &  0 &  0 &  0 \\
        52 & 36 & 21 & 21 & 12 &  9 &  8 &  3 &  3 &  2 &  3 &  2 &  0 &  0 &  0 &  0 \\
        \bottomrule
    \end{tabular}

	}
	\label{tab:lifetime}
\end{table}
\vfill\null
