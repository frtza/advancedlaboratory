\section*{Anhang}

\addcontentsline{toc}{section}{Anhang}

\centering

\null\vfill
\begin{table}[H]
	\centering
	\caption{Messwerte zu Verzögerungszeit und Zählrate. Aus poissonverteilter Zählung wird für
			 den Fehler die Quadratwurzel der Gesamtzahl gebildet und dann durch das Messintervall
			 $\qty{10}{\second}$ dividiert.}
	    \begin{tabular}{S[table-format=2.0] S[table-format=3.0]}
        \toprule
        {$t \mathbin{/} \unit{\nano\second}$} & {$N$} \\
        \midrule
        -30 &   0 \\
        -29 &   0 \\
        -28 &   0 \\
        -27 &   0 \\
        -26 &   0 \\
        -25 &   0 \\
        -24 &   0 \\
        -23 &   0 \\
        -22 &   0 \\
        -21 &   0 \\
        -20 &   0 \\
        -19 &   0 \\
        -18 &   0 \\
        -17 &   0 \\
        -16 &   0 \\
        -15 &   0 \\
        -14 &   0 \\
        -13 &   0 \\
        -12 &   0 \\
        -11 &   1 \\
        -10 &   2 \\
         -9 &   5 \\
         -8 &   8 \\
         -7 &   8 \\
         -6 &   9 \\
         -5 &  10 \\
         -4 &  13 \\
         -3 &  14 \\
         -2 &  16 \\
         -1 &  20 \\
          0 &  20 \\
          1 &  18 \\
          2 &  22 \\
          3 &  18 \\
          4 &  20 \\
          5 &  20 \\
          6 &  16 \\
          7 &  19 \\
          8 &  14 \\
          9 &  15 \\
         10 &  11 \\
         11 &  10 \\
         12 &  10 \\
         13 &   6 \\
         14 &   5 \\
         15 &   4 \\
         16 &   1 \\
         17 &   1 \\
         18 &   0 \\
         19 &   0 \\
         20 &   1 \\
         21 &   0 \\
         22 &   0 \\
         23 &   0 \\
         24 &   0 \\
         25 &   0 \\
         26 &   0 \\
         27 &   0 \\
         28 &   0 \\
         29 &   0 \\
         30 &   0 \\
        \bottomrule
    \end{tabular}

	\label{tab:delay}
\end{table}
\vfill\null

\newpage

\null\vfill
\begin{table}[H]
	\centering
	\caption{Messwerte zu Kanalindex, Zeitintervall und Zählrate. Einzelne Einträge entsprechen jeweils einer
			 Laufzeit von $\qty{10}{\second}$ am Doppelpulsgenerator.}
	    \begin{tabular}{S[table-format=3.0] S[table-format=1.1]}
        \toprule
        {$K$} & {$t \mathbin{/} \unit{\micro\second}$} \\
        \midrule
          4 & 0.5 \\
         16 & 1.0 \\
         38 & 2.0 \\
         61 & 3.0 \\
         84 & 4.0 \\
        107 & 5.0 \\
        129 & 6.0 \\
        152 & 7.0 \\
        175 & 8.0 \\
        197 & 9.0 \\
        218 & 9.9 \\
        \bottomrule
    \end{tabular}

	\label{tab:calibration}
\end{table}
\vfill\null

\newpage

\null\vfill
\begin{table}[H]
	\centering
	\caption{Zählraten sortiert nach unten und rechts mit aufsteigender Kanalnummer. In der Messzeit
			 $\protect\qty{158234}{\second}$ wurden $\protect\num{4509112}$ Startpulse und
			 $\protect\num{17526}$ Stopppulse registriert. Die nachfolgenden Werte beziehen
			 sich auf dieselbe Messzeit.}
	\noindent\makebox[\linewidth]{
	    \sisetup{table-format=3.0}
    \begin{tabular}{S S S S S S S S S S S S S S S S}
        \toprule
        \multicolumn{16}{c}{$N$} \\
        \midrule
          0 &  80 &  49 &  26 &   7 &   9 &   6 &   4 &   0 &   0 &   0 &   0 &   0 &   0 &   0 &   0 \\
          0 &  86 &  47 &  25 &   8 &   9 &   5 &   2 &   0 &   0 &   0 &   0 &   0 &   0 &   0 &   0 \\
          0 &  78 &  52 &  28 &  15 &  11 &   5 &   0 &   0 &   0 &   0 &   0 &   0 &   0 &   0 &   0 \\
          0 &  76 &  42 &  18 &  12 &   8 &   5 &   0 &   0 &   0 &   0 &   0 &   0 &   0 &   0 &   0 \\
        113 &  96 &  37 &  16 &   9 &   5 &   1 &   0 &   0 &   0 &   0 &   0 &   0 &   0 &   0 &   0 \\
        141 &  65 &  41 &  23 &   8 &   6 &   9 &   0 &   0 &   0 &   0 &   0 &   0 &   0 &   0 &   0 \\
        144 &  67 &  43 &  19 &  14 &   5 &   7 &   0 &   0 &   0 &   0 &   0 &   0 &   0 &   0 &   0 \\
        158 &  72 &  38 &  19 &  14 &   7 &   8 &   0 &   0 &   0 &   0 &   0 &   0 &   0 &   0 &   0 \\
        143 &  69 &  47 &  19 &  17 &   8 &   7 &   0 &   0 &   0 &   0 &   0 &   0 &   0 &   0 &   1 \\
        128 &  76 &  35 &  16 &  11 &   6 &   2 &   0 &   0 &   0 &   0 &   0 &   0 &   0 &   0 &   0 \\
        130 &  72 &  34 &  22 &   7 &   5 &   9 &   0 &   0 &   0 &   0 &   0 &   0 &   0 &   0 &   0 \\
        163 &  64 &  42 &  15 &   7 &   8 &   4 &   0 &   0 &   0 &   0 &   0 &   0 &   0 &   0 &   0 \\
        128 &  55 &  39 &  14 &  12 &   7 &   3 &   0 &   0 &   0 &   0 &   0 &   0 &   0 &   0 &   0 \\
        123 &  63 &  33 &  18 &  11 &   9 &   2 &   0 &   0 &   0 &   0 &   0 &   0 &   0 &   0 &   0 \\
        110 &  51 &  31 &  19 &   9 &   3 &   6 &   0 &   0 &   0 &   0 &   0 &   0 &   0 &   0 &   0 \\
        135 &  62 &  26 &  13 &   6 &   3 &   4 &   0 &   0 &   0 &   0 &   0 &   0 &   0 &   0 &   0 \\
        128 &  68 &  30 &  17 &  13 &   6 &   4 &   0 &   0 &   0 &   0 &   0 &   0 &   0 &   0 &   0 \\
        119 &  51 &  36 &  14 &   6 &   9 &   5 &   0 &   0 &   0 &   0 &   0 &   0 &   0 &   0 &   0 \\
        111 &  55 &  30 &  23 &   7 &   7 &   7 &   0 &   0 &   0 &   0 &   0 &   0 &   0 &   0 &   0 \\
        105 &  47 &  34 &  19 &   7 &   6 &   2 &   0 &   0 &   0 &   0 &   0 &   0 &   0 &   0 &   0 \\
        104 &  46 &  32 &  14 &  13 &   2 &   4 &   0 &   0 &   0 &   0 &   0 &   0 &   0 &   0 &   0 \\
        104 &  54 &  26 &  14 &  13 &   5 &   4 &   0 &   0 &   0 &   0 &   0 &   0 &   0 &   0 &   0 \\
        104 &  63 &  35 &  14 &   8 &   6 &   3 &   0 &   0 &   0 &   0 &   0 &   0 &   0 &   0 &   0 \\
         93 &  52 &  32 &  23 &  12 &   7 &   5 &   0 &   0 &   0 &   0 &   0 &   0 &   0 &   0 &   0 \\
        116 &  56 &  30 &  13 &   4 &   4 &   7 &   0 &   0 &   0 &   0 &   0 &   0 &   0 &   0 &   0 \\
         89 &  50 &  31 &  14 &   8 &   4 &   5 &   0 &   0 &   0 &   0 &   0 &   0 &   0 &   0 &   0 \\
         95 &  50 &  32 &  11 &   3 &   2 &   1 &   0 &   0 &   0 &   0 &   0 &   0 &   0 &   0 &   0 \\
         84 &  47 &  26 &  16 &   5 &   8 &   0 &   0 &   0 &   0 &   0 &   0 &   0 &   0 &   0 &   0 \\
         86 &  48 &  16 &  17 &   7 &   6 &   3 &   0 &   0 &   0 &   0 &   1 &   0 &   0 &   0 &   0 \\
         75 &  47 &  22 &  14 &  11 &   3 &   5 &   0 &   0 &   0 &   0 &   0 &   0 &   0 &   0 &   0 \\
         95 &  42 &  23 &  20 &   8 &   6 &   3 &   0 &   0 &   0 &   0 &   0 &   0 &   0 &   0 &   0 \\
         73 &  45 &  19 &  12 &   7 &   5 &   5 &   0 &   0 &   0 &   0 &   0 &   0 &   0 &   0 &   0 \\
        \bottomrule
    \end{tabular}

	}
	\label{tab:lifetime}
\end{table}
\vfill\null
