\section{Diskussion}

Zusammenfassend lässt sich festellen, dass mit den bisher genannten Werten plausible Ergebnisse
vorliegen. Sowohl für die Einstellung der Messapparatur als auch die eigentliche Messung sind die
Daten mit der Erwartung vereinbar. Im folgenden wird auf die jeweiligen Messabschnitte einzeln
eingegangen.



\subsection{Einstellung}

Der in Abbildung \eqref{fig:delay} gezeigte Zählratenverlauf lässt sich als Trapez mit Halbwertsbreite
$\qty{17.1+-0.3}{\nano\second}$ nähern, dessen Plateau bei $\num{199+-12}$ in etwa dem Sollwert
$\qty{20}{\per\second}$ entspricht. Bei der Einstellung der optimalen Verzögerungszeit ist es in diesem
Fall egal, ob die Mitte des Plateaus $\qty{6.8+-0.7}{\nano\second}$ oder alternativ die Mitte der Halbwertsbreite
$\qty{2.7+-0.2}{\nano\second}$ verwendet werden, da die Skala nur halbzahlige Schritte zulässt.

Damit und unter Berücksichtigung des Endergebnisses scheint sich die zuvor aufgestellte Annahme, dass die
erhöhte Zählrate am Abgriff des zweiten Photomultipliers ab der Koinzidenzschaltung einen vernachlässigbaren
Fehlereinfluss darstellt, zu bestätigen. Ob der Ursprung dieser Abweichung am PMT
selbst oder im Diskriminator liegt, kann nicht abschließend geklärt werden. Gegen einen fehlerhaften
Diskriminator spricht, dass am selben Ausgang unter Austauschen des ersten gegen den zweiten
Photomultiplier die besagte höhere Zählrate beobachtet werden kann. Außerdem ist diese bereits
am Oszilloskop ohne zwischengeschalteten Diskriminator zu erkennen. Andererseits stellt sich das
Einstellen des Potentiometers am Diskriminator für den zweiten PMT als schwierig
heraus, da dieses teilweise keinen Effekt zu haben scheint, was wiederum auf einen Fehler des
Bauteils hindeuten könnte.

Bei der Kalibrierung des Vielkanalanalysators wird durch den Verlauf in Abbildung \eqref{fig:calibration}
sowie den geringen Fehler der Fitparameter ein linearer Zusammenhang zwischen Kanal und entsprechendem
Zeitintervall bestätigt. Die Zählraten in Tabelle \eqref{tab:calibration} belegen zudem eine uniforme Effizienz
über den relevanten Messbereich.



\subsection{Messung}

Die Langzeitmessung zur Bestimmung der Lebensdauer in Abbildung \eqref{fig:lifetime} folgt mit dem
unmodifizierten Binning des Vielkanalanalysators in guter Näherung der angesetzten Exponentialverteilung.
Unter Ausschluss solcher Wertebereiche, die über mehrere Kanäle Null ergeben, folgt für Myonen
$\qty{1.03+-0.03}{\micro\second}$ als mittlere Zerfallszeit, deren Fehlerbereich den Literaturwert
$\qty{2.197}{\micro\second}$ \cite{Tishchenko_2013} beinhaltet und die somit als Ergebnis akzeptiert wird.

Anhand der Ausgleichrechnung gilt für das gegebene Messintervall von $\qty{154196}{\second}$ eine
Hintergrundrate $\input{build/m.tex}$ durch fälscherlicherweise als Stopppuls gewertetes Eintreten weiterer
Myonen. Diese ist ebenso mit dem Wert $\num{2.5}$ aus der Poissonstatistik vereinbar.

Zuletzt kann noch eine weitere Struktur in den Daten aus Tabelle \eqref{tab:lifetime} vermutet werden. Dabei
handelt es sich um einen Peak im Bereich $\qty{0.5}{\micro\second}$ mit bis zu $\num{100}$ als Zählrate.
Für dieses Phänomen gibt es zunächst keine offensichtliche Erklärung, wobei erst durch bessere Statistik
aus längerer Messung geklärt werden müsste, ob dies überhaupt ein echter Effekt ist oder eine zufällige
Schwankung repräsentiert.
