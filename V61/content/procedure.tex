\section{Procedure}

\section{Procedure}

\subsection{Aligning the Laser}

The alignment of the He-Ne laser is initiated by positioning an auxiliary alignment laser (wavelength: 532 nm, maximum power: 1 mW, reduced power: 0.2 mW) on the optical bench. A target screen with a cross mark is placed at the end of the optical rail. The alignment laser should be adjusted such that its beam passes through the center of the cross on the target screen. This ensures that the alignment laser coincides with the optical axis of the He-Ne laser.

Next, the He-Ne laser components are positioned in the following order: the laser tube (length: 408 mm, diameter: 1.1 mm), resonator mirrors (diameter: 12.7 mm), and Brewster windows. These components together form the laser resonator, with the Brewster windows ensuring minimal loss while defining the polarization direction. The alignment of these components is critical: the alignment laser's back reflections should be made to hit the target screen's cross at the center, indicating proper alignment along the optical axis.

\subsection{Verifying the Stability Condition}

Once the laser has been aligned, the stability condition must be verified. The laser is adjusted to its maximum power using a photodiode, and the maximum resonator length is set by gradually increasing the gap between the two resonator mirrors. Throughout the process, the laser power is continuously readjusted. With a well-aligned setup, the system should approach the theoretical value from the stability condition. This step is repeated for different resonator lengths to study the effect on the laser's stability.

\subsection{Observing Transverse Modes}

To observe transverse electromagnetic (TEM) modes, a thin tungsten wire (diameter: 0.005 mm) is placed between the resonator mirror and the laser tube. This wire stabilizes different modes, which can be observed on an optical screen. A scattering lens may be used to enlarge the laser beam, making it easier to identify the modes. The wire functions to stabilize the laser beam, enabling clearer mode identification. The optical screen is then replaced with a photodiode to measure the intensity distribution for at least two modes. The measured intensity distributions are plotted and compared with theoretical expectations to validate the mode stability.

\subsection{Determining the Polarization}

The laser's polarization is determined by placing a polarizer behind the outcoupling mirror. The intensity of the laser beam is measured with a photodiode as the polarizer is rotated. The Brewster windows, which minimize reflection losses, ensure a well-defined polarization direction. By comparing the experimental intensity distribution with theoretical calculations, the polarization characteristics of the laser are evaluated, and the influence of the Brewster windows and resonator mirrors on the polarization is examined.

\subsection{Analyzing Spectra in Multimode Operation}

In the absence of a Fabry-Perot etalon, the laser operates in multimode, meaning several longitudinal modes coexist. This leads to temporal intensity variations due to the beating between modes. To analyze these beat frequencies, a fast photodiode with a bandwidth up to 1 GHz is used, and the Fourier spectra are recorded for various resonator lengths with a spectrum analyzer. The spread of the neon transition is compared with the distance between the longitudinal modes, and the multimode operation is justified. Additionally, the dependence of the beat frequency on the resonator length is investigated.

\subsection{Measuring the Wavelength}

The wavelength of the He-Ne laser is determined by using diffraction patterns produced by a slit and diffraction grating. The diffraction maxima and minima are measured to accurately determine the wavelength. This method provides a precise measurement of the laser's wavelength and can be used to verify the laser's output characteristics.
