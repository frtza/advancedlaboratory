% \newpage
\subsection{Measuring the Wavelength}

To calculate the wavelength, one finds
\begin{equation*}
	\lambda = \frac{g x_n}{n \sqrt{d^2 + x_n^2}} \: ,
\end{equation*}
with the distance $x_n$ between the maximum of order $n$ and the main peak, the slit width $g$ and the spacing $d$ of the screen.

\begin{table}
	\centering
	\caption{Measured diffraction peak distances on a screen at $d = \qty{83.2+-1.0}{\centi\meter}$ from different gratings
			 with calculated wavelengths.}
	\sisetup{table-format=2.1}
	\begin{tabular}{c S S S S}
		\toprule
		& \multicolumn{2}{c}{$g = 1/80$ mm} & \multicolumn{2}{c}{$g = 1/100$ mm} \\
		\cmidrule(lr){2-3}\cmidrule(lr){4-5}
		{$n$} & {$x_{n, \text{left}}$ / cm} & {$x_{n, \text{right}}$ / cm} & {$x_{n, \text{left}}$ / cm} & {$x_{n, \text{right}}$ / cm} \\
		\midrule
		1 &  4.5 &  4.3 &  5.4 &  5.4 \\
		2 &  8.7 &  8.4 & 11.1 & 10.7 \\
		3 & 12.8 & 12.9 & 16.5 & 16.1 \\
		4 & 17.1 & 17.3 & 22.2 & 21.8 \\
		5 & 21.6 & 22.2 &  &  \\
		\cmidrule(lr){2-3}\cmidrule(lr){4-5}
		& \multicolumn{2}{c}{$\lambda = \qty{641+-10}{\nano\meter}$} &
		\multicolumn{2}{c}{$\lambda = \qty{644+-9}{\nano\meter}$} \\
		\cmidrule(lr){2-5}
		& \multicolumn{4}{c}{$\lambda = \qty{643+-12}{\nano\meter}$} \\
		\bottomrule
	\end{tabular}
	\label{tab:wavelength}
\end{table}

Listed in Table \ref{tab:wavelength} are the measurements with their corresponding results. The derived wavelength
$\lambda = \qty{643+-12}{\nano\meter}$ can be compared to the value $\lambda = \qty{633}{\nano\meter}$ from the
literature \cite{Eichler_2018}.



\section{Discussion}

Summarizing the obtained results, all theoretical expectations are fulfilled within the given tolerances. Due to limitations in the extend
of the setup as well as difficulties in alignment, especially for the plane konvex resonator configuration, the expected intensity drop when
reaching the stability limits could not be observed, though the measured lasing range is still in agreement with the allowed $L$ intervals.
The function given by the respective Hermite polynomials and a Gaussian distribution describes the observed TEM profile reasonably well,
though measuring steps could have tighter spacing to get a more conclusive picture. The angular dependence of the polarizer fits the theory
very well, giving a minimum intensity at $\qty{155}{\degree}$ compatible with zero, thus confirming the laser beam to be fully polarized thanks
to the Brewster window. Similarly, the dependence of frequency maxima position on the reciprocal resonator length can also be verified, while
the Doppler broadening seems to represent the dominant influence on the line width. It should be noted, however, that the bandwidth of the
photodiode itself leads to a similar decay and makes this measurement fairly unreliable, likely leading to an underestimation of the effect.
Lastly, the wavelength calculated via diffraction maxima agrees with the expected value from the involved transition.
